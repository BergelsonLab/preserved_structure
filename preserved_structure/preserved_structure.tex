% Template for Cogsci submission with R Markdown

% Stuff changed from original Markdown PLOS Template
\documentclass[10pt, letterpaper]{article}

\usepackage{cogsci}
\usepackage{pslatex}
\usepackage{float}
\usepackage{caption}

% amsmath package, useful for mathematical formulas
\usepackage{amsmath}

% amssymb package, useful for mathematical symbols
\usepackage{amssymb}

% hyperref package, useful for hyperlinks
\usepackage{hyperref}

% graphicx package, useful for including eps and pdf graphics
% include graphics with the command \includegraphics
\usepackage{graphicx}

% Sweave(-like)
\usepackage{fancyvrb}
\DefineVerbatimEnvironment{Sinput}{Verbatim}{fontshape=sl}
\DefineVerbatimEnvironment{Soutput}{Verbatim}{}
\DefineVerbatimEnvironment{Scode}{Verbatim}{fontshape=sl}
\newenvironment{Schunk}{}{}
\DefineVerbatimEnvironment{Code}{Verbatim}{}
\DefineVerbatimEnvironment{CodeInput}{Verbatim}{fontshape=sl}
\DefineVerbatimEnvironment{CodeOutput}{Verbatim}{}
\newenvironment{CodeChunk}{}{}

% cite package, to clean up citations in the main text. Do not remove.
\usepackage{cite}

\usepackage{color}

% Use doublespacing - comment out for single spacing
%\usepackage{setspace}
%\doublespacing


% % Text layout
% \topmargin 0.0cm
% \oddsidemargin 0.5cm
% \evensidemargin 0.5cm
% \textwidth 16cm
% \textheight 21cm

\title{Preserved Structure Across Vector Space Representations}

\usepackage{lipsum}
\usepackage[utf8]{inputenc}
\usepackage[export]{adjustbox}

\author{{\large \bf Andrei Amatuni, Elika Bergelson} \\ \texttt{\{andrei.amatuni, elika.bergelson\}@duke.edu} \\ 417 Chapel Dr. Durham, NC 27708 USA \\ Department of Psychology and Neuroscience \\ Duke University}

\begin{document}

\maketitle

\begin{abstract}
Certain concepts, words, and images are intuitively more similar than
others (dog vs.~cat, dog vs.~spoon), though quantifying such similarity
is notoriously tricky. Indeed, this kind of computation is likely a
critical part of learning the category boundaries for words within a
given language. Here, we use a set of items (n= 27, e.g. `dog') that are
highly common in young infants'input, and use both image-based and
word-based algorithms to independently compute similarity among these
items. We find two key results. First, the pair-wise item similarities
derived within image-space and word co-occurrence-space are correlated,
suggesting evidence of preserved structure among these extremely
different representational formats. Second, the `closest' neighbors for
each item computed within each space overlap significantly. We further
discuss the potential role of animacy in a set of exploratory analyses.
We conclude that this approach, which does not rely on human ratings of
similarity, may nevertheless reflect stable within-class structure
across these two spaces. We speculate that such invariance, in turn,
might aid in lexical acquisition, by serving as an informative marker of
category boundaries.

\textbf{Keywords:}
vector space models; semantic similarity; word learning
\end{abstract}

\section{Introduction}\label{introduction}

Infants are presented with a challenge to carve the world into distinct
lexical entities in the process of learning their first language.
They're provided with little supervision while mapping a territory which
William James (1890) dubbed a ``great blooming, buzzing confusion''. How
they determine which aspects of the world to attend to in service of
this goal, is an area of ongoing research (Mareschal \& Quinn, 2001).
Different features of objects and their environments are varyingly
informative with regards to object segmentation and category structure.
Some researchers have suggested that categorization is along
fundamentally perceptual grounds and that only later in development is
conceptual knowledge incorporated into these nascent perceptual
categories (Quinn \& Eimas, 1997, 2000; Quinn, Johnson, Mareschal,
Rakison, \& Younger, 2000). Others suggest that there are in fact two
distinct processes at work, such that perceptual categories are computed
automatically by the sensory systems, while conceptual categories are
independently formed through conscious action (Mandler, 2000). Träuble
and Pauen (2007) provide evidence of functional information (regarding
the animacy of objects) influencing early category judgements. Gelman
and Markman (1986) explicitly set these two sources of category cues
against each other (i.e.~functional vs.~perceptual), and found that
preschoolers can override perceptual overlap in reasoning about
functional similarity, in natural kinds.

The degree to which conceptual and perceptual information are separable,
both in early learning, and in adult experts, is an important open
question. Any model which hopes to explain the mechanics of human
categorization must address how these potentially disparate
information-sources interface in mental representations, and to what
degree they interact. While evidence from human learners suggests they
integrate perceptual and linguistic information during categorization
and learning (\textbf{REFS}), here we take first steps in a deliberately
different approach. We separate computations over images and words, and
then comparing the overlap in the similarity among items that these
systems deduce. Using a set of highly familiar and common words and
concepts from a large infant corpus, we compare the output of an image
analysis, and a word co-occurrence analysis for these same items. We use
algorithms that learn feature representations without hand engineering,
purely as a byproduct of their particular (and separate) training
objectives (e.g.~natural language processing or object recognition in
images). Comparing the representations these algorithms learn, we gain a
window into the structure of visual and semantic forms.

\section{Methods}\label{methods}

\subsection{Items}\label{items}

The 27 items (i.e.~words and images) analyzed here were selected due to
their high frequency in infants' early visual and linguistic input,
aggregated as part of the SEEDLingS project, which gathered longitudinal
audio and video data of infants' home environments from 6-17 months
(Bergelson, 2016a, 2016b). We describe this larger study in brief,
simply to relay how the items we analyze were chosen; the details of
this larger study are not critical to the present work. In the larger
study, 44 infants were tested every other month (from 6-18 months) on
common nouns, using a looking-while-listening eyetracking design in
which two images are shown on a screen, and one is named. The words for
these experiments were chosen by dint of being high frequency or well
known across infants in other samples (e.g.~Brent Corpus, WordBank
\textbf{ADD REFS}), or by being one of the top 10 concrete nouns heard
in each infants' own home audio and video recordings in the preceding
two months.

The images displayed when these words were tested were chosen from a
library of prototypical images of a given word (e.g.~dog), along with
images of infants' own versions of the items, as seen in their home
videos (e.g.~a given infants' cat, specific bottle, etc.). To enter the
current analysis, images had to occur 9 or more times in this image
library of high frequency concrete nouns \textbf{add M(SD) R for images
and words in allbasiclevel}. Thus, the words and images used here were
highly common across our 500+ daylong audio-recordings and 500+ hourlong
video-recordings from 44 infants, thus providing an ecologicallyvalid
item-set for present modeling purposes.

The images for these 27 items used to derive category vectors in the
image analysis below were all 960x960pixel images of a single object on
a gray background. All items correspond to words found on WordBank
(Frank, Braginsky, Yurovsky, \& Marchman, 2017), a compilation of the
MacArthur-Bates Communicative Development Inventory, which we use as a
proxy for age of acquisition below \textbf{add ref fenson \& dale}.

\subsection{Vector Representations}\label{vector-representations}

We generate two sets of vector representations for a common set of words
first learned by most infants. The first set of vectors are taken from a
pretrained set of GloVe representations (Pennington, Socher, \& Manning,
2014), a modern distributional semantic vector space model. The second
set is taken from the final layer activations of a pretrained image
recognition model, Google's Inception V3 convolutional neural network
(Szegedy, Vanhoucke, Ioffe, Shlens, \& Wojna, 2016). Both of these
representations are what's refered to as ``embeddings''. They map
objects from one medium (e.g.~images or words) into a metric space where
distances between points can be computed and function as a measure of
similarity between objects.

\subsubsection{Word Vectors}\label{word-vectors}

\subsubsection{Image Vectors}\label{image-vectors}

In the case of our word vectors, the GloVe algorithm instantiates the
distributional hypothesis, which proposes that words which co-occur with
each other share similar meaning (Firth, 1957; Harris, 1954), and by
capturing the covariance of tokens in large text corpora, you capture
some aspect of their semantic structure. The image embeddings, on the
other hand, are taken from the final layer of activations in a
convolutional neural network, whose objective function tunes network
parameters in service of object recognition, where the loss function is
computed in reference to a set of labeled training images (Russakovsky
et al., 2015). The final layer of this network encodes the most abstract
and integrated visual features, serving as the basis for classification
into 1000 different classes.

\subsection{Defining a prototypical
image}\label{defining-a-prototypical-image}

In the case of word vectors, each word is assigned a unique point in a
common vector space. Different images containing objects of the same
type, on the other hand, will have varying vector representations after
passing through the layers of a neural network. This presents a problem
in comparing the two forms of representation. We must first define the
most prototypical (or average) image vector for any given category of
object.

Given a set of images \(S_c\) containing objects belonging to a single
category \(c\) (e.g.~cat, dog, chair), we define our prototypical vector
\(\hat{x}_c\) of \(S_c\) as the generalized median within a
representational space \(U\). This is the vector with minimal sum of
distances between it and all the other members of set \(S_c\) in \(U\).
If \(x\) and \(y\) are vectors in space \(U\), products of images in
\(S_c\) being passed through a neural network, then

\[
 \hat{x_c} = \operatorname*{arg\,min}_{x\in U} \sum_{y\in U} d(x, y)
\] We define our \(d(x, y)\) to be the cosine similarity measure:

\[
d(x, y) = 1 - \frac{x\cdot y}{\|x\|\|y\|}
\]

Our \(d(x, y)\) is not a metric in the strict sense, but is less
susceptible to differences in \(L^2\) norm influencing our measure of
similarity, as is the case with the Euclidean distance. These
differences in magnitude between vectors can be the product of frequency
effects in the training data, and the cosine similarity corrects for
this.

\subsection{Comparing spaces}\label{comparing-spaces}

After we have our two sets of vectors (i.e.~those from word vector space
and those from image vector space), we can compare all the pairwise
distances between objects, both within a single space and across the
two. When comparing across the two spaces, a correlation in pairwise
distances implies that inter-object distances have been conserved. For
example, if ``dog'' and ``cat'' are close together in word space and
mutually far apart from ``chair'' and ``table'' in that same space,
maintaining this relationship for all pairwise distances in the
\textit{other} vector space means that the global inter-object structure
is preserved across this mapping, despite being in radically different
spaces, both in terms of dimensionality (300 for words, and 2048 for
images in our case) and by virtue of using completely different
algorithms and inputs to establish the vector representations for
objects. So while their absolute locations might have been radically
transformed, this correlation would be a measure of the
\textit{degree of invariance} in their positioning relative to each
other.

\section{Results}\label{results}

To test whether image- and word-based similarity converged for this set
of common items, we conducted several analyses. First, we tested whether
the pairwise cosine distances for all items in word vector space
correlated with those same pairwise distances in the image vector space
(see Figure \ref{fig:pairwise-corr}). Indeed, we find a significant
correlation (\(R = 0.30\), \(p < 1.5e-08\))
\footnote{we present Pearson correlations above in keeping with the linear fits on the graph; all reported significance patterns are identical using Spearman's $\rho$.}

Next, we examined the degree to which our set of 27 words shared
overlapping neighbors in the two vector spaces (see Table
\ref{tbl:overlap-table}). We defined `neighbor' by first determining the
mean similarity between each item and the 26 other items. Any items
whose distance to this 'target \textbf{say something clear here} was
considered a neighbor. With this normalized neighborhood threshold, we
find that the majority of items have at least 1 neighbor which is shared
across representational spaces. Within word-space, items had on average
XX neighbors, Range XX-XX. Within Image-space, items had XX neighbors,
Range XX-XX.

If we compute neighbor `overlap' across the spaces (i.e.~how many of the
neighbors overlapped, divided by how many neighbors there were) we find
that the overlap is significantly greater than 0 (Mean Overlap XX,
p\textless{} XX by Wilcoxon test). This complements the correlational
analysis, showing not just that the distances for any given pair tended
to have similar values in image-space and word-space, but that most
similar words/images for each of the 27 items also were consistent
across these spaces.

\subsubsection{Animacy (Eb stopped here)}\label{animacy-eb-stopped-here}

If we partition the set of inter-word distances into those that are
either animate-animate, inanimate-inanimate, or mixed, we find that the
pairs of distances between inanimate objects significantly correlate
across our two spaces (\(R = 0.38\), \(p < 0.00024\)), while the other
two pairings do not (see Figure \ref{fig:pairwise-corr-animate-vs-not}).

We expect that those classes of objects which preserve their structure
between representations more strongly would result in earlier
object-referent mappings. This is because inferences about
object-referent mappings conditioned on both visual and semantic
features would be more stable compared to those cases where the two
representations vary independently. For example, an object that is both
round (i.e.~visual feature) and tends to roll (i.e.~semantic feature)
would be more salient as a distinct entity than an object whose visual
features are entirely uninformative about its functional or semantic
qualities.

In our current analysis the class of objects which displays stronger
structure preservation (within class) are the inanimate objects. When we
partition our set of 27 words into animates and inanimates and plot
their relative AoA, we find a noticeable though insignificant preference
for inanimates, as expected (see Figures
\ref{fig:animacy-aoa-prod-graph} and \ref{fig:animacy-aoa-comp-graph}).
The choice to partition our set into these two categories is to a degree
arbitrary, and we have no reason to believe infants would learn one
class of objects earlier than the other. Our current analysis is offered
purely as an exploratory exercise, suggesting that perhaps partitions
along other taxonomic or associative lines may provide insight in future
investigations.

\begin{CodeChunk}
\begin{figure}[tb]
\includegraphics{figs/pairwise-corr-1} \caption[Relative cosine distance between points in word embedding space correlates with relative distance in image embedding space ($R = 0.30$, $p < 1.5e-08$)]{Relative cosine distance between points in word embedding space correlates with relative distance in image embedding space ($R = 0.30$, $p < 1.5e-08$). Graph contains all pairwise distances for every word.}\label{fig:pairwise-corr}
\end{figure}
\end{CodeChunk}

\begin{CodeChunk}
\begin{figure}[tb]
\includegraphics{figs/pairwise-corr-animate-vs-not-1} \caption[Inanimate objects display a significantly stronger correlation when mapping across vector spaces, meaning that they preserve their within-class structural relationships more reliabily across these two spaces]{Inanimate objects display a significantly stronger correlation when mapping across vector spaces, meaning that they preserve their within-class structural relationships more reliabily across these two spaces. Animate and mixed distances do not correlate. Each graph contains all pairwise distances between objects that are either a) both animate ($R = -0.13$, $p < 0.27$), b) both inanimate ($R = 0.38$, $p < 0.00024$), or c) mixed animate-to-inanimate ($R = -0.01$, $p < 0.86$)}\label{fig:pairwise-corr-animate-vs-not}
\end{figure}
\end{CodeChunk}

\begin{table}
\centering
\includegraphics[max size={\columnwidth}{0.7\textheight}]{data/overlap_table_formatted.png}
\caption{Overlaps between closest objects in image vector space and word vector space. Neighbors are defined as those other objects which are less than -1 SD from the mean distance for any given word. Those neighbors that are marked red are shared between image and vector spaces. The overlap ratio is the number of shared neighbors across vector spaces divided by the total unique neighbors between the two spaces.}
\label{tbl:overlap-table}
\end{table}

\begin{CodeChunk}
\begin{figure}[tb]
\includegraphics{figs/animacy-aoa-prod-graph-1} \caption[AoA for animates vs inanimates (using child production data) collapsed over month]{AoA for animates vs inanimates (using child production data) collapsed over month}\label{fig:animacy-aoa-prod-graph}
\end{figure}
\end{CodeChunk}

\begin{CodeChunk}
\begin{figure}[tb]
\includegraphics{figs/animacy-aoa-comp-graph-1} \caption[AoA for animates vs inanimates (using child comprehension data) collapsed over month]{AoA for animates vs inanimates (using child comprehension data) collapsed over month}\label{fig:animacy-aoa-comp-graph}
\end{figure}
\end{CodeChunk}

\section{Discussion}\label{discussion}

We've reported a significant correspondence between representations
learned by two different algorithms operating over seemingly unrelated
inputs (i.e.~visual and linguistic). What is most noteworthy here is
that the only immediate common ground between these representations are
the real life objects they both aim to model. This draws us into
questions concerning the nature of similarity and the multifaceted
character of information which is revealed by objects in the real world.
The notion that we can make inferences about one aspect of an object
given another aspect, is not surprising or controversial. However, the
fact that we can make these bi-directional inferences using aspects
traditionally treated as being orthogonal, is noteworthy. This is
particularly the case given the enormous dimensionality of our feature
spaces, and the fact that these algorithms are placed under no pressure
to find homologous representations.

Through what metrics can a learning algorithm, or indeed a human,
establish gradations of likeness? Are these necessarily the same metrics
which form the basis of category boundaries? These are fundamental
questions which have enjoyed a long history in the field (Edelman, 1998;
Hahn, Chater, \& Richardson, 2003; Kemp, Bernstein, \& Tenenbaum, 2005;
Shepard \& Chipman, 1970; Tversky, 1977). While our current work is not
sufficient to support a specific mechanism responsible for the observed
regularity, it might be indicative of the special role of invariance,
given that the unifying thread between our algorithms and inputs are the
common objects they represent. Underneath the diversity of visual
statistics and token distributions lie stable entities in the world
which, by virtue of their invariant actuality, give rise to regularity
across measurements at different vantage points (i.e.~modalities), an
idea dating back to Helmholtz (1878).

We find in our current work that this quality of invariance is
differentially present across different classes of entities, namely
animate vs.~inanimate objects. However, this is conditioned on the
particular algorithms we've investigated here, and our extensions into
human performance with our AoA anlysis did not show a significant
sensitivity to this difference. This could suggest a number things. The
first is that humans might not discover the regularities that these
algorithms do. Or it could be that our current class partitioning does
not provide sufficient contrast in invariance to register human AoA
differences. Or it could be that regularity is not a determining factor
in ease of acquisition. Of these three, the last is least likely to be
the case.

\section{Conclusion}\label{conclusion}

We find evidence of an interaction between visual and semantic features
learned by two distinct machine learning algorithms which operate over
drastically different inputs, and are trained in the service of
seemingly unrelated ends. This interaction is indicative of conserved
structure between these two supposedly independent sources of
information (i.e.~visual and functional). If humans are sensitive to
this relationship, as these algorithms seem to be, we expect that those
classes of object which are more strongly invariant across feature
spaces would be more easily learned by infants. We find a noticeable
though insignificant relationship between this property and AoA in our
current partitioning scheme (animates vs.~inanimates).

\section{Acknowledgements}\label{acknowledgements}

We thank the SEEDLingS team, and NIH DP5-OD019812.

\section{References}\label{references}

\setlength{\parindent}{-0.1in} \setlength{\leftskip}{0.125in} \noindent

\hypertarget{refs}{}
\hypertarget{ref-bergelson2016seedlings}{}
Bergelson, E. (2016a). Bergelson seedlings homebank corpus.
\url{http://doi.org/10.21415/T5PK6D}

\hypertarget{ref-bergelson2016seedlingsdatabrary}{}
Bergelson, E. (2016b). SEEDLingS corpus. Retrieved January 26, 2018,
from \url{https://nyu.databrary.org/volume/228}

\hypertarget{ref-edelman1998representation}{}
Edelman, S. (1998). Representation is representation of similarities.
\emph{Behavioral and Brain Sciences}, \emph{21}(4), 449--467.

\hypertarget{ref-firth1957synopsis}{}
Firth, J. R. (1957). A synopsis of linguistic theory, 1930-1955.
\emph{Studies in Linguistic Analysis}.

\hypertarget{ref-frank2017wordbank}{}
Frank, M. C., Braginsky, M., Yurovsky, D., \& Marchman, V. A. (2017).
Wordbank: An open repository for developmental vocabulary data.
\emph{Journal of Child Language}, \emph{44}(3), 677--694.

\hypertarget{ref-gelman1986categories}{}
Gelman, S. A., \& Markman, E. M. (1986). Categories and induction in
young children. \emph{Cognition}, \emph{23}(3), 183--209.

\hypertarget{ref-hahn2003similarity}{}
Hahn, U., Chater, N., \& Richardson, L. B. (2003). Similarity as
transformation. \emph{Cognition}, \emph{87}(1), 1--32.

\hypertarget{ref-harris1954distributional}{}
Harris, Z. S. (1954). Distributional structure. \emph{Word},
\emph{10}(2-3), 146--162.

\hypertarget{ref-helmholtz1878facts}{}
Helmholtz, H. (1878). The facts of perception. \emph{Selected Writings
of Hermann Helmholtz}, 1--15.

\hypertarget{ref-james2013principles}{}
James, W. (1890). \emph{The principles of psychology}. Henry Holt;
Company.

\hypertarget{ref-kemp2005generative}{}
Kemp, C., Bernstein, A., \& Tenenbaum, J. B. (2005). A generative theory
of similarity. In \emph{Proceedings of the 27th annual conference of the
cognitive science society} (pp. 1132--1137).

\hypertarget{ref-mandler2000perceptual}{}
Mandler, J. M. (2000). Perceptual and conceptual processes in infancy.
\emph{Journal of Cognition and Development}, \emph{1}(1), 3--36.

\hypertarget{ref-mareschal2001categorization}{}
Mareschal, D., \& Quinn, P. C. (2001). Categorization in infancy.
\emph{Trends in Cognitive Sciences}, \emph{5}(10), 443--450.

\hypertarget{ref-pennington2014glove}{}
Pennington, J., Socher, R., \& Manning, C. (2014). Glove: Global vectors
for word representation. In \emph{Proceedings of the 2014 conference on
empirical methods in natural language processing (emnlp)} (pp.
1532--1543).

\hypertarget{ref-quinn1997reexamination}{}
Quinn, P. C., \& Eimas, P. D. (1997). A reexamination of the
perceptual-to-conceptual shift in mental representations. \emph{Review
of General Psychology}, \emph{1}(3), 271.

\hypertarget{ref-quinn2000emergence}{}
Quinn, P. C., \& Eimas, P. D. (2000). The emergence of category
representations during infancy: Are separate perceptual and conceptual
processes required? \emph{Journal of Cognition and Development},
\emph{1}(1), 55--61.

\hypertarget{ref-quinn2000understanding}{}
Quinn, P. C., Johnson, M. H., Mareschal, D., Rakison, D. H., \& Younger,
B. A. (2000). Understanding early categorization: One process or two?
\emph{Infancy}, \emph{1}(1), 111--122.

\hypertarget{ref-ILSVRC15}{}
Russakovsky, O., Deng, J., Su, H., Krause, J., Satheesh, S., Ma, S.,
\ldots{} Fei-Fei, L. (2015). ImageNet Large Scale Visual Recognition
Challenge. \emph{International Journal of Computer Vision (IJCV)},
\emph{115}(3), 211--252. \url{http://doi.org/10.1007/s11263-015-0816-y}

\hypertarget{ref-shepard1970second}{}
Shepard, R. N., \& Chipman, S. (1970). Second-order isomorphism of
internal representations: Shapes of states. \emph{Cognitive Psychology},
\emph{1}(1), 1--17.

\hypertarget{ref-szegedy2016rethinking}{}
Szegedy, C., Vanhoucke, V., Ioffe, S., Shlens, J., \& Wojna, Z. (2016).
Rethinking the inception architecture for computer vision. In
\emph{Proceedings of the ieee conference on computer vision and pattern
recognition} (pp. 2818--2826).

\hypertarget{ref-trauble2007role}{}
Träuble, B., \& Pauen, S. (2007). The role of functional information for
infant categorization. \emph{Cognition}, \emph{105}(2), 362--379.

\hypertarget{ref-tversky1977features}{}
Tversky, A. (1977). Features of similarity. \emph{Psychological Review},
\emph{84}(4), 327.

\end{document}
